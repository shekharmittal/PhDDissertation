%%%%%%%%%%%%%%%%%%%%%%%%%%%%%%%%%%%%%%%%%%%%%%%%%%%%%%%%%%%%%%%%%%%%%%%%
%                                                                      %
%                          PRELIMINARY PAGES                           %
%                                                                      %
%%%%%%%%%%%%%%%%%%%%%%%%%%%%%%%%%%%%%%%%%%%%%%%%%%%%%%%%%%%%%%%%%%%%%%%%

\title{Essays on Taxation in Emerging Economies}
\author{Shekhar Mittal}
\department{Management}
\degreeyear{2018}



%%%%%%%%%%%%%%%%%%%%%1%%%%%%%%%%%%%%%%%%%%%%%%%%%%%%%%%%%%%%%%%%%%%%%%%%%

\abstract       {My dissertation contributes towards our understanding of firms behavior in weakly institutionalized environments. It consists of three chapters. The first, ``Detecting Auctioneer Corruption: Evidence from Russian Procurement Auctions'' develops a novel method for detecting auctioneer corruption in first-price sealed-bid auctions. I study the leakage of bid information by the auctioneer to a preferred bidder. I construct a formal test for the presence of bid-leakage cor11ruption and apply it to a novel data set of $4.3$ million procurement auctions in Russia that occurred between 2011 and 2016. With bid leakage, the preferred bidder gathers information on other bids and waits until the end of the auction to place a bid. Such behavior creates an abnormal correlation between winning and being (chronologically) the last bidder. Informed by this fact, I build several measures of corruption. I document that more than $10\%$ of the auctions were affected by bid leakage. My results imply that the value of the contracts assigned through these auctions was $\$1.2$ billion over the six-year study period. I build a model of bidding behavior to show that corruption exerts two effects on the expected prices of the contracts. The direct effect inflates the price of the contract. The indirect effect reduces the expected price since honest bidders are trying to undercut corrupt bidders. I find both effects in the data, with the direct effect being more pronounced.

My second chapter, ``Collusion in Auctions: Evidence from the Timing of Bids'' documents collusion between firms using a unique feature of the same Russian procurement data: the timestamps of all bids. Timestamp data allows developing a new method of collusion detection based on the excessive share of simultaneous bids. My method shows that 8--23\% of winner-runner-up pairs bid together, which provides a bound on the share of collusive auctions. Next, I document that simultaneous bidding is correlated with higher procurement prices and smaller bid margins in the auctions. We include a battery of controls to state that collusion leads to 8--9\% increase in the final price of the contracts and makes joint bids up to 50\% closer to each other. The chapter is the first to show how one can enhance methods of collusion detection by using the data on timing of bids.

In the third and last chapter, I study the effects of armed conflict on trade transactions between firms. The chapter examines trade in the aftermath of the Russian-Ukrainian conflict (2014–). The geographic concentration of fighting in a few regions allows me to study the indirect effects of conflict on trade, as opposed to the direct effects of violence or trade embargoes. I employ a highly granular transaction-level dataset for the universe of import and export transactions in Ukraine and find that firms from more ethnolinguistically Ukrainian counties experienced a deeper drop in trade with Russia relative to the firms in more Russian counties. The richness of panel data allows looking beyond explanations not related to ethnicities, such as increased transportation costs and bans on certain products. Instead, I focus on two ethnic-specific explanations: a rise in animosity and a decrease in trust. In a stylized model of trade with asymmetric information, I show that one can distinguish these two mechanisms based on whether the effect is more pronounced for homogeneous or non-homogeneous goods, the latter pointing to the trust mechanism. The intuition is that trust mitigates the uncertainty behind goods' quality. Empirically, I show that, in contrast to homogeneous goods, the trade of relation-specific goods have not changed differentially across ethnic lines. Hence, I find little evidence in support of a shock to trust. I then use survey data to show that inter-ethnic animosity has indeed escalated in the aftermath of the conflict. 

}

%%%%%%%%%%%%%%%%%%%%%%%%%%%%%%%%%%%%%%%%%%%%%%%%%%%%%%%%%%%%%%%%%%%%%%%%

\chair         {Romain T. Wacziarg}
\chair          {Aprajit Mahajan}
\member           {Nico Voigtl{\"a}nder}
\member         {Ricardo Perez-Truglia}
\member         {Adriana Lleras-Muney}
\member         {Paola Giuliano}


%%%%%%%%%%%%%%%%%%%%%%%%%%%%%%%%%%%%%%%%%%%%%%%%%%%%%%%%%%%%%%%%%%%%%%%%

\dedication     {To Nana, my first chearleader. To everyone who took a chance on me.}

%%%%%%%%%%%%%%%%%%%%%%%%%%%%%%%%%%%%%%%%%%%%%%%%%%%%%%%%%%%%%%%%%%%%%%%%

\acknowledgments{I am especially indebted to my Committee Chairs Aprajit Mahajan  and Romain Wacziarg, for their invaluable support and guidance throughout my PhD studies. I am extremely thankful for the support and advice from my other committee members John Asker and Christian Dippel. I am also grateful to Leonardo Bursztyn, Lorenzo Casaburi, Ernesto Dal Bo, Georgy Egorov, Stefano Fiorin, Gaston Illanes, Seema Jayachandran, Kei Kawai, Lidia Kosenkova, Imil Nurutdinov, Ricardo Perez-Truglia, Nicola Persico, Robert Porter, Nancy Qian, Noam Yuchtman, and the participants of the Brown Bag at the Anderson School of Management, DEVPEC2016, GEM-BPP workshop, RSSIA2016, NEUDC2016, Northwestern IO lunch, IIOC2017, MWIEDC2017, and TADC2017, and seminar participants at the University of Wisconsin-Madison, New Economic School, and CERGE-EI Prague for helpful comments and suggestions. I gratefully acknowledge financial support from the UCLA Anderson Center for Global Management. All remaining errors are mine. The first chapter is a version of a working paper with the same title co-authored with Pasha Andreyanov and Alec Davidson (both from UCLA). The third chapter is a version of a working paper with the same title co-authored with Alexey Makarin from Northwestern in preparation for publication. For both co-authored chapters, Vasily and other co-authors were co-PIs.

This study would not have been possible without the support provided by officials in the Department of Trade and Taxes of the Government of NCT of Delhi. We thank Sanjiv Sahai, Principal Secretary, Department of Finance for his undiminishing encouragement. We also thank Pierre Bachas, Michael Best, Joshua Blumenstock, Christian Dippel, Stefano Fiorin, Paola Giuliano, Adriana Lleras-Muney, Ricardo Perez-Truglia, Dina Pomeranz, Nico Voigtlander, Roman Wacziarg, and the anonymous referees for their valuable comments and helpful suggestions. 

%The work is supported by the \grantsponsor{}{International Growth Center (India)}{} under  Grant No.: ~\grantnum{}{89412}, \grantsponsor{}{Economic Development and Institutions Grant}{} under  Grant No.: ~\grantnum{}{A0014-19809} and \grantsponsor{}{JPAL-GI}{}. We also acknowledge support and assistance from \grantsponsor{}{CEGA (UC Berkeley)}{}.

}


%\begin{comment}
%%%%%%%%%%%%%%%%%%%%%%%%%%%%%%%%%%%%%%%%%%%%%%%%%%%%%%%%%%%%%%%%%%%%%%%%
\vitaitem { \textbf{Education} } {}
\vitaitem {2013} {MPP, Harris School of Public Policy, Chicago}
\vitaitem {2007} {Bachelor of Engineering, Computers, Delhi University}
\vitaitem { \textbf{Awards} } {}

\vitaitem {2017-2018} {UCLA Dissertation Year Fellowship}
\vitaitem {2016} {Russell Sage Foundation Award}
\vitaitem {2013-2016} {Center for Global Management Grants \vspace{0.8cm}} 
\vitaitem {2013-2017} {UCLA Anderson Fellowship\vspace{0.8cm}}

%%%%%%%%%%%%%%%%%%%%%%%%%%%%%%%%%%%%%%%%%%%%%%%%%%%%%%%%%%%%%%%%%%%%%%%%
%\end{comment}


%%%%%%%%%%%%%%%%%%%%%%%%%%%%%%%%%%%%%%%%%%%%%%%%%%%%%%%%%%%%%%%%%%%%%%%%
%%%%%%%%%%%%%%%%%%%%%%%%%%%%%%%%%%%%%%%%%%%%%%%%%%%%%%%%%%%%%%%%%%%%%%%%
%                                                                      %
%                          PRELIMINARY PAGES                           %
%                                                                      %
%%%%%%%%%%%%%%%%%%%%%%%%%%%%%%%%%%%%%%%%%%%%%%%%%%%%%%%%%%%%%%%%%%%%%%%%

\title{Essays on Taxation in Emerging Economies}
\author{Shekhar Mittal}
\department{Management}
\degreeyear{2018}

%%%%%%%%%%%%%%%%%%%%%1%%%%%%%%%%%%%%%%%%%%%%%%%%%%%%%%%%%%%%%%%%%%%%%%%%%
\abstract  {Even though compliance issues are central to taxation policies in emerging economies, convincing empirical research on tax compliance has been scarce. Through the three chapters of my dissertation, I bridge this gap by using detailed value added tax (VAT) micro-data from Delhi, India. 

A key advantage of VAT type systems is that they allow for corroboration of transactions using returns of interacting firms. In chapter 1, co-author Aprajit Mahajan and I evaluate the effect of a technology reform that improved the Delhi tax authority’s ability to cross-check buyer reports against seller reports within the VAT system. Before the technology change, such cross-checks could only be accomplished by auditing both parties, a relatively rare and time-consuming activity. After the policy change, the tax authority could (and did) relatively easily cross-check information, declared by buyers with the corresponding information from sellers, directly on its own servers without initiating an audit. We use a difference-in-difference approach to show that the policy had a large and significant effect on wholesalers relative to retailers. A wholesaler is more likely to sell to registered firms whereas a retailer is more likely to sell to final customers where the paper trail breaks down. Therefore, on the output side, the self-enforcing mechanism of the VAT is more likely to break down for retailers compared to wholesalers. We also find significant heterogeneity with almost the entire increase being driven by changes in the behavior of the largest tax-paying firms. This result sheds light on limits of third-party verification in a context with limited audit resources and where the majority of firms do not remit any tax. 

In low compliance environments, a common strategy to manipulate the third-party verification system is to establish fraudulent (``bogus'') firms. Bogus firms help genuine firms in reducing their tax burden by issuing fake receipts. A tax authority determines the existence of bogus firms by first filtering down based on a few preliminary indicators, and then undertaking  physical inspections. Given the authority’s limited resources, these inspections are only done sporadically. A key challenge in improving tax compliance then is to regularly, cheaply and reliably identify such bogus firms. In chapter 2, coauthors Aprajit Mahajan, Ofir Reich and I apply a machine learning classifier to the same tax dataset to identify bogus firms which can be further targeted for physical inspections. We face a nonstandard applied machine learning scenario. First, one-sided labels: firms that are not caught as bogus are of unknown class: bogus or legitimate, and we need to not only use them to train the classifier but also make predictions on them. Second, multiple time-periods: each firm files several periodic VAT returns but its class is fixed so prediction needs to be made at the firm, not firm-period, level. Third, point in time simulation: we estimate the revenue saving potential of our model by simulating the implementation of our system in the past. We do this by rolling back the data to the state of knowledge at a specific time and calculating the revenue impact of acting on our model’s recommendations and catching the bogus firms and estimate US\$40 million in recovered revenue.

Tax authorities commonly apply size based regulations to firms. If firms are concerned about compliance costs, then such regulations create adverse incentives for firms to stay small. These regulations also increase the monitoring effort needed from tax officials. In the first two years of our dataset, the Delhi tax system had multiple turnover based filing frequency thresholds. Firms with declared turnover (in the previous year) less than \rupee 1 million had to file returns annually, between \rupee 1 and 5 million - semiannually, between \rupee 5 and 50 million - quarterly, and more than \rupee 50 million - monthly. In the years 3, 4 and 5 of our dataset, this turnover based filing policy was first weakened and then completely disbanded. In chapter 3, coauthor Jan Luksic and I first show that this policy resulted in bunching of firms below the thresholds at all levels. Using the change in these reporting policies, we provide further evidence that such sharp bunching indeed occurs due to the VAT reporting frequency thresholds. Second, we calculate the VAT revenue losses due to such bunching and document the longer-term impact of the VAT reporting frequency thresholds. Finally, the subsequent withdrawal of the policy allows us to show that in a regime with size-dependent reporting requirements, more frequent reporting does not lead to greater levels of VAT collection.
}

%%%%%%%%%%%%%%%%%%%%%%%%%%%%%%%%%%%%%%%%%%%%%%%%%%%%%%%%%%%%%%%%%%%%%%%%
\chair         {Romain T. Wacziarg}
\chair          {Aprajit Mahajan}
\member           {Nico Voigtl{\"a}nder}
\member         {Ricardo Nico Perez Truglia}
\member         {Adriana Lleras-Muney}
\member         {Paola Giuliano}
%%%%%%%%%%%%%%%%%%%%%%%%%%%%%%%%%%%%%%%%%%%%%%%%%%%%%%%%%%%%%%%%%%%%%%%%

\dedication     {To Nana, my biggest champion.}

%%%%%%%%%%%%%%%%%%%%%%%%%%%%%%%%%%%%%%%%%%%%%%%%%%%%%%%%%%%%%%%%%%%%%%%%
\acknowledgments{
I thank my advisors Romain Wacziarg and Aprajit Mahajan who are not only inspiring researchers but also approachable and caring mentors. Romain Wacziarg has given me the opportunity to explore my own agenda, even when I was clearly struggling.  Many times, especially during my coursework stumbles, I have found solace in his kind and encouraging words. He has gone out of his way to provide me with financial support, and has always been ready to reach out to others on my behalf. Aprajit Mahajan has provided the much needed calm to my excitable personality. He has provided me with a safety net while I have tried to push for ambitious collaborations. He has taught me the value of integrity, rigor and reworking the draft.  Most importantly, both of them have given me the freedom to fail outstandingly.

I would also like to thank my committee members - Paola Giuliano, Adriana Lleras-Muney, Ricardo Perez-Truglia, and Nico Voigtl{\"a}nder. Without their advice and support I could not have reached here. Paola Giuliano has more faith in my ability than me, and has always been ready to provide inputs from the ``other'' side. In no other class have I been as motivated to do well as in the population economics course taught by Adriana Lleras-Muney. Ricardo Perez Truglia's research and guidance has helped me keep my work grounded in public finance challenges. Nico Voigtl{\"a}nder has always been insightful and constructive, not to mention kind.  

This dissertation is a culmination of (or a milestone in) a journey in which many people have contributed in unexpected ways and it is impossible to name and thank them all. I am grateful to both Marianne Bertrand and Pablo Montagnes, who started me on my research journey and who were unexpectedly patient with an enthusiastic master's student at University of Chicago. In my initial PhD years, I have greatly benefited from long conversations with Christian Dippel. I have also greatly benefitted from discussions with and advice from Pierre Bachas, Prabhat Barnwal, Youssef Benzarti, Michael Best, Josh Blumenstock, Andrea DiMiceli, Stefano Fiorin, Vasily Korovkin, Aparna Krishnan, Temina Madon, Atishi Marlena, Dina Pomeranz, Jasmine Shah and Divya Singh. I thank them for their generosity. I also thank participants at RIDGE Conference-Public Economics (2017), the GEM seminar (Anderson School, UCLA), GEM-BPP workshop (2017), Center for Business Taxation Doctoral Meeting (Oxford, 2017), APPAM Conference (Chicago, 2017), NEUDC Conference (Tufts, 2017), and 2nd Zurich Conference on Public Finance in Developing Countries (2017). All remaining errors are mine. 

Chapter 1 is a version of a working paper \cite{mittal2017vat}, and is currently being prepared for publication. I thank coauthor Aprajit Mahajan for his contribution to the study design and the writing. Chapter 2 is the final version of \cite{mittal2018bogus}, which is set to appear in the proceedings of the 1st Annual ACM SIGCAS Conference on Computing and Sustainable Societies, COMPASS 2018. I thank coauthors Aprajit Mahajan and Ofir Reich for their contribution to the study design, data analysis and writing. I have and continue to benefit from Ofir Reich's friendship as well as his wisdom on challenges in applied machine learning. Chapter 3 is a work in progress with the same title coauthored with Jan Luksic. I thank Jan Luksic for his contribution to the literature review, to the analytical strategy, and to the writing. I feel blessed to have worked with such coauthors.

This dissertation would not have been possible without the generous time, access to data, and answers to questions provided by officials in the Department of Trade and Taxes of the Government of National Capital Territory of Delhi. I thank Sanjiv Sahai, Principal Secretary, Department of Finance for his undiminishing support. The work in this disseration was funded with UK Aid from the UK Government and grants from JPAL Governance Initiative. I am also thankful for the support that has been provided by JPAL South Asia, International Growth Center (India), and University of California Center for Effective Global Action.

I thank my friends Neha Dar, Shipra Gupta, Rahul Jha, Pranesh Nagarajan, Megha Punjani, Deepak Rajanna, Parag Taneja, and Manjari Vishnoi for letting me be stupid around them. I am guilty of taking them for granted and from each of them I have gotten more than I have deserved. I am grateful for the continuing love and support of my family. Without the support of my chacha-chachi, Mahesh Mittal and Jyotsna Mittal, this journey would have never begun. My brother Anant Mittal is my 1 man entourage. My parents, Kailash Mittal and Sarika Mittal, have always provided me with more freedom and trust than I have deserved and have tried their best to not bias me with their own priors. 

Finally, all this will not be worth it without my most ardent critic, my fiercest supporter, and my partner in crime, Shiwali Mohan by my side. She is my soul food. I lucked out that she settled for me.}
%%%%%%%%%%%%%%%%%%%%%%%%%%%%%%%%%%%%%%%%%%%%%%%%%%%%%%%%%%%%%%%%%%%%%%%%
\vitaitem { \textbf{Education} } {}
\vitaitem {2013} {MPP, Harris School of Public Policy, Chicago}
\vitaitem {2007} {Bachelor of Engineering, Computers, Delhi University}
\vitaitem { \textbf{Awards} } {}
\vitaitem {2017-2018} {UCLA Dissertation Year Fellowship}
\vitaitem {2013-2017} {UCLA Anderson Fellowship\vspace{0.8cm}}

\publication{\textbf{Shekhar Mittal}, Ofir Reich and Aprajit Mahajan. \emph{Who is Bogus? Using One-Sided Labels to Identify Fraudulent Firms from Tax Returns}. In Proceedings of the 1st Annual ACM SIGCAS Conference on Computing and Sustainable Societies, COMPASS 2018. doi: https://doi.org/10.1145/3209811.3209824}

\vitaitem { \textbf{Presentations} } {}
\vitaitem {2018} {PacDev*; $2^{nd}$ International Conf. on Globalization and Development*; PMRC (NUS)*; \footnotesize{* indicates declined}}
\vitaitem {2017} { Office of Chief Economic Advisor (Govt of India); GEM-BPP Workshop; RIDGE Public Economics (Uruguay); Center for Business Taxation Doctoral Meeting (Oxford); APPAM Fall Research Conference (Chicago); NEUDC Conference 2017 (Tufts); $2^{nd}$ Zurich Conference on Public Finance in Developing Countries;}
\vitaitem{2016}{ Dept of Finance (Govt of NCT of Delhi); Graduate Student Brownbag (UC-Riverside);}
\vitaitem{\textbf{Grants}}{}
\vitaitem{2018}{\textbf{J-PAL Governance Initiative (\$50,000)}. Improving State Response to Public Grievances.}
\vitaitem{2017}{\textbf{J-PAL Governance Initiative (\$7,500)}. Improving the Efficacy of Public Procurement and Public Grievance Monitoring.}
\vitaitem{2017}{\textbf{EDI consortium (\pounds 22,000)}. Who is Bogus? Catching fraudulent firms in Delhi.}
\vitaitem{2016}{\textbf{IGC Research Grant (\pounds 50,830)}. Where's Value? Using VAT data to Improve Compliance.}
\vitaitem{2015}{\textbf{J-PAL Governance Initiative (\$49,050)}. Information Provision and Participatory Budgeting: Mohalla Sabhas in Delhi.}
\vitaitem{2015}{\textbf{J-PAL Governance Initiative (\$5,000)}. Improving Public Service via the Ballot Box: Evidence from Delhi.}
%%%%%%%%%%%%%%%%%%%%%%%%%%%%%%%%%%%%%%%%%%%%%%%%%%%%%%%%%%%%%%%%%%%%%%%%